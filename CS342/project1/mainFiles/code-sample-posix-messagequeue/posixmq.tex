\section{POSIX Message Queues}

Processes can communicate with each other by using message queues in
Linux. You can use the man pages to learn about the message queue
interface (POSIX message queue API). Just type: {\tt man mq\_overview}
on your shell and read the related help page.

Here, we provide an example application that uses a message queue
between two processes to pass information. The application is the
producer-consumer application. The producer will produce items that
will be passed to the consumer which will consume the items. The items
will be passed through the message queue.

The consumer program will create the message queue. Hence, it should
be run first. The producer process, when started, will open the
message queue that was created earlier. 

The two programs can share some definitions. We put them into a header
file called {\tt shareddefs.h}. This header file is shown below. It
includes the definition of a C structure that will represent an item.
It also includes the name of the message queue that will be
created. That name can be any valid file name (starting with
``/''). Producer and consumer programs will use that name to refer to
the same message queue.

\VerbatimInput{posixmq/shareddefs.h}

Below, we show the producer program. It opens the message queue by
calling the {\tt mq\_open} system call (a library function in fact,
which in turn calls the respective system call) Then, in a while loop,
it generates and sends messages into the message queue. To send a
message, we use the {\tt mq\_send} function.  Between two send
operations we call the sleep function to sleep for a while.

\VerbatimInput{posixmq/producer.c}

Next, we have the program for the consumer process.

\VerbatimInput{posixmq/consumer.c}

The consumer program first creates a message queue with the name
specified in the shareddefs.h header file. This name is shared by the
producer and consumer processes. We use the {\tt mq\_open} function to
create a message queue as well.  After creating the message queue, the
consumer learns about some of the properties of the message queue by
calling the mq\_getattr() function. In this case we are learning the
maximum size of a message that kernel can support. Then we allocate
that much space for our local buffer to put an incoming item (message)
from the producer.  Our local buffer is just a character (byte) array
pointed by bufptr. We use the malloc() function to allocate memory to
be used as the buffer.

We can compile these programs and obtain the respective executable
files by using the following Makefile.

\VerbatimInput{posixmq/Makefile}

After editing such a Makefile, we just need to type {\tt make} to
compile both of the programs and obtain two executable files, i.e.,
programs: {\tt producer} and {\tt consumer}. Then, in one window, you
can run the producer program and in another window you can run the
consumer program. To run the consumer, you can just type the following
in one window: {\tt consumer} or {\tt ./consumer}

The consumer will create a message queue and will wait on it for the
arrival of a message.

To run the producer, we can type the following in another window: {\tt
  producer} or {\tt ./producer}

Producer will start generating items every one second and will send
them into message queue. 

The consumer process will start running and retrieving the items from
the message queue. It will print out the content of each item to the
screen. Producer and consumer processes will run indefinitely until
you terminate them (by a program like the {\tt kill} command). You can
type {\tt kill -9 pid} to kill a process whose process id is pid.


Note that in the Makefile while compiling the programs we use an
option {\tt -lrt}. That means we have to link our program with the
real-time library, {\tt librt}. The is because the message queue
related API functions are implemented in the rt library.
